\chapter{前言}
\renewcommand{\baselinestretch}{10.0} %設定行距
\pagenumbering{arabic} %設定頁號阿拉伯數字
\setcounter{page}{1}  %設定頁數
\fontsize{14pt}{2.5pt}\sectionef
\section{整體架構}
    經過討論後,我們將整個BubbleRob分為三大主題,分別為 「 SolidWorks 」 、「 CoppeliaSim 」 、「 Lua 」,再分數個小主題去完成本次競賽。下圖為本次競賽整體架構(圖.\ref{fig.機器人報告表格})\\

\begin{figure}[hbt!]
\begin{center}
\includegraphics[width=13cm]{機器人報告表格}
\caption{\Large 心智圖}\label{fig.機器人報告表格}
\end{center}
\end{figure}


\section{規則說明}
類似於手足球,一開始時球會置於場中央,遊戲開始後兩方在不同地點連上同一個網路即可
以鍵盤操控機器人推球至已方的球門得分。\\
 
遊戲規則如下:\\

足球觸碰到球門感測器即算得分\\
在限時內得越多分數的隊伍即獲勝\\
任一方進球得分後,球員的位置不變,球會回到場中央重新開始\\
\begin{figure}[hbt!]
\begin{center}
\includegraphics[width=15cm]{機器人報告2}
\caption{\Large 機器人報告2 }
\label{機器人報告2 }
\end{center}
\end{figure}
\section{未來展望}
此專題希望能利用現有完成的機械學習的算法,能發展成虛擬訓練,再將訓練完的機器學習應用到虛擬環境或是實體機電系統,並透過伺服器將影像串流提供玩家網頁介面進行遠端操控,同時提供多人觀看及時的比賽影像,將整個冰球機的控制和使用者間有更完善串聯,機電系統的部分達到最優化控制和虛實整合的應用。
\section{規則說明}
 Pong game 的遊戲規則簡單,透過擊錘將球打入對方球門即得一分,只要其中一方得21分就結束該局。擊錘只能沿單方向來回移動來進行防守和進攻。\\
遊戲規則如下:
\begin{enumerate}
\item 球打入敵方即得一分。
\item 擊錘只單一方向移動。
\item 最快贏得21分者獲勝,並結束該局遊戲。
\end{enumerate}

\renewcommand{\baselinestretch}{0.5} %設定行距